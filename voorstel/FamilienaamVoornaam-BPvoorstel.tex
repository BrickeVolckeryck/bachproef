%==============================================================================
% Sjabloon onderzoeksvoorstel bachproef
%==============================================================================
% Gebaseerd op document class `hogent-article'
% zie <https://github.com/HoGentTIN/latex-hogent-article>

% Voor een voorstel in het Engels: voeg de documentclass-optie [english] toe.
% Let op: kan enkel na toestemming van de bachelorproefcoördinator!
\documentclass{hogent-article}

% Invoegen bibliografiebestand
\addbibresource{voorstel.bib}

% Informatie over de opleiding, het vak en soort opdracht
\studyprogramme{Professionele bachelor toegepaste informatica}
\course{Bachelorproef}
\assignmenttype{Onderzoeksvoorstel}
% Voor een voorstel in het Engels, haal de volgende 3 regels uit commentaar
% \studyprogramme{Bachelor of applied information technology}
% \course{Bachelor thesis}
% \assignmenttype{Research proposal}

\academicyear{2025-2026} % TODO: pas het academiejaar aan

% TODO: Werktitel
\title{Het ontwikkelen van een geïntegreerd framework dat de veiligheid garandeert en volledige traceerbaarheid biedt voor alle interacties tussen gebruikers en een digitale assistent in klantondersteuningsprocessen}

% TODO: Studentnaam en emailadres invullen
\author{Bricke Volckeryck}
\email{bricke.volckeryck@student.hogent.be}

% TODO: Medestudent
% Gaat het om een bachelorproef in samenwerking met een student in een andere
% opleiding? Geef dan de naam en emailadres hier
% \author{Yasmine Alaoui (naam opleiding)}
% \email{yasmine.alaoui@student.hogent.be}

% TODO: Geef de co-promotor op
\supervisor[Co-promotor]{K. Haeck (Evolane, \href{mailto:kristof.haeck@evolane.eu}{kristof.haeck@evolane.eu})}

% Binnen welke specialisatierichting uit 3TI situeert dit onderzoek zich?
% Kies uit deze lijst:
%
% - Mobile \& Enterprise development
% - AI \& Data Engineering
% - Functional \& Business Analysis
% - System \& Network Administrator
% - Mainframe Expert
% - Als het onderzoek niet past binnen een van deze domeinen specifieer je deze
%   zelf
%
\specialisation{System \& Network Administrator}
\keywords{AI Security, Observability, Chatbot Interactions, Akamai AI Firewall, Dynatrace Integration}

\begin{document}

\begin{abstract}
  In moderne digitale klantomgevingen worden AI-chatbots steeds vaker ingezet om ondersteuning te bieden, informatie te verstrekken en transacties te begeleiden. Deze groeiende afhankelijkheid van geautomatiseerde interacties brengt echter risico’s met zich mee, onder meer op het vlak van datalekken, manipulatie via prompt-injecties en een gebrek aan inzicht in wat er precies tijdens een gesprek gebeurt. Daardoor ontstaat een duidelijke nood aan een oplossing die zowel traceerbaarheid als beveiliging kan garanderen, en tegelijk kan voldoen aan privacy- en GDPR-vereisten.
  \\
  
  Dit onderzoek heeft als doel een geïntegreerd framework te ontwikkelen dat de volledige communicatiestroom tussen eindgebruikers en een AI-chatbot inzichtelijk en beveiligd maakt. Hiervoor wordt onderzocht hoe Dynatrace kan worden ingezet om gesprekslogs, acties en systeeminteracties in real time te monitoren, en hoe Akamai’s AI Web \& API Firewall verdachte input, aanvallen of ongewenste datastromen kan detecteren en tegenhouden. De centrale onderzoeksvraag luidt op welke manier deze twee technologieën gecombineerd kunnen worden om een traceerbare én veilige communicatiestroom te realiseren.
  \\
  
  Om deze vraag te beantwoorden wordt een prototype-chatbot ontwikkeld binnen een eenvoudig klantondersteuningsscenario. Deze chatbot wordt geïntegreerd met Dynatrace voor gedetailleerde logging, tracing en auditability, terwijl Akamai’s AI Firewall wordt geconfigureerd om inkomend en uitgaand verkeer te scannen en te beveiligen. Vervolgens worden de verzamelde observability-gegevens en security-events met elkaar vergeleken om inzicht te krijgen in aanvalspatronen, afwijkend gedrag en mogelijke risico’s.
  \\
  
  Het onderzoek verwacht aan te tonen dat de combinatie van observability en AI-gedreven beveiliging een aanzienlijk hoger niveau van controle, transparantie en incidentrespons biedt dan klassieke monitoring- of firewalloplossingen afzonderlijk. Bovendien kan het voorgestelde framework organisaties ondersteunen bij het naleven van privacy- en complianceverplichtingen, doordat elke interactie reproduceerbaar en controleerbaar wordt vastgelegd. Hierdoor vormt dit werk een waardevolle stap richting veilige, betrouwbare en juridisch verdedigbare AI-gestuurde klantcommunicatie.
\end{abstract}

\tableofcontents

% De hoofdtekst van het voorstel zit in een apart bestand, zodat het makkelijk
% kan opgenomen worden in de bijlagen van de bachelorproef zelf.
%---------- Inleiding ---------------------------------------------------------

% TODO: Is dit voorstel gebaseerd op een paper van Research Methods die je
% vorig jaar hebt ingediend? Heb je daarbij eventueel samengewerkt met een
% andere student?
% Zo ja, haal dan de tekst hieronder uit commentaar en pas aan.

%\paragraph{Opmerking}

% Dit voorstel is gebaseerd op het onderzoeksvoorstel dat werd geschreven in het
% kader van het vak Research Methods dat ik (vorig/dit) academiejaar heb
% uitgewerkt (met medesturent VOORNAAM NAAM als mede-auteur).
% 

\section{Inleiding}%
\label{sec:inleiding}

\noindent De inzet van digitale assistenten binnen klantondersteuning neemt sterk toe binnen professionele IT-omgevingen. Organisaties gebruiken dergelijke systemen om klantvragen sneller te beantwoorden, incidenten op te volgen en administratieve processen te automatiseren. In deze context worden echter vaak gevoelige gegevens verwerkt, zoals persoonsgegevens, bestelgegevens of interne bedrijfsinformatie. Binnen veel organisaties ontbreekt vandaag een sluitend mechanisme om deze interacties volledig te volgen, te beveiligen en achteraf te reconstrueren. Dit vormt een concreet probleem wanneer beveiligingsincidenten optreden of wanneer naleving van privacywetgeving moet worden aangetoond.\\

\noindent Deze bachelorproef vertrekt vanuit een concrete bedrijfscontext bij Evolane, een IT-dienstverlener die organisaties ondersteunt op het vlak van observability en security. Binnen deze casus stelt zich de uitdaging om AI-gestuurde chatbotinteracties niet alleen operationeel betrouwbaar te laten functioneren, maar ook volledig traceerbaar en beveiligd te maken. Meer specifiek ontbreekt vandaag een geïntegreerde aanpak waarbij zowel de inhoud van gesprekken als beveiligingsgebeurtenissen op een samenhangende manier worden geregistreerd, gecorreleerd en geanalyseerd. Dit gebrek aan inzicht bemoeilijkt incidentanalyse, auditing en het aantonen van compliance.\\

\noindent De doelgroep van deze bachelorproef bestaat uit IT- en securityprofessionals binnen Evolane en vergelijkbare organisaties, die verantwoordelijk zijn voor het beheren, beveiligen en monitoren van digitale klantinteracties.\\

\noindent Vanuit deze probleemsituatie wordt volgende centrale onderzoeksvraag geformuleerd:
Hoe kan een geïntegreerd framework op basis van Dynatraces Audit Logging en Akamai’s AI Firewall worden ontworpen om AI-chatbotinteracties volledig traceerbaar en beveiligd te maken, met behoud van privacy en naleving van GDPR?\\

\noindent Om deze onderzoeksvraag te beantwoorden, richt het onderzoek zich op het combineren van observability en security binnen een samenhangend framework. Daarbij wordt onderzocht hoe interacties technisch gelogd en getraceerd kunnen worden, hoe verdachte of risicovolle invoer kan worden gedetecteerd en hoe deze informatie kan worden gecorreleerd om incidenten efficiënt te analyseren en te mitigeren.\\

\noindent De onderzoeksdoelstelling van deze bachelorproef is het ontwerpen en realiseren van een proof-of-concept dat aantoont hoe een dergelijke geïntegreerde oplossing in de praktijk kan functioneren. Het concrete eindresultaat bestaat uit een werkend prototype van een digitale assistent, aangevuld met een observability- en securitylaag die interacties inzichtelijk en controleerbaar maakt. De bachelorproef kan als geslaagd worden beschouwd wanneer het prototype aantoont dat chatbotinteracties op een reproduceerbare, veilige en privacybewuste manier kunnen worden gemonitord en beveiligd.\\

%---------- Stand van zaken ---------------------------------------------------

\section{Literatuurstudie}%
\label{sec:literatuurstudie}

\noindent De opkomst van artificiële intelligentie binnen klantgerichte toepassingen heeft geleid tot een sterke toename van geautomatiseerde interacties tussen eindgebruikers en digitale systemen. In het bijzonder AI-chatbots worden steeds vaker ingezet voor klantondersteuning, verkoop en incidentopvolging. Hoewel deze technologie duidelijke voordelen biedt op vlak van efficiëntie en schaalbaarheid, Introduceert dit tegelijk ook nieuwe uitdagingen op het gebied van beveiliging, traceerbaarheid en privacy. Deze literatuurstudie beschrijft de huidige stand van zaken binnen dit domein en benadrukt de nood aan geïntegreerde oplossingen die observability en beveiliging combineren.

\subsection{AI-chatbots in bedrijfscontext}

\noindent AI-chatbots maken gebruik van natuurlijke taalverwerking en machine learning om menselijke interacties te simuleren en gebruikersvragen te beantwoorden. Volgens \textcite{Adamopoulou2020} worden dergelijke systemen vooral ingezet om repetitieve taken te automatiseren en de werkdruk op menselijke medewerkers te verlagen. In bedrijfscontexten verwerken chatbots echter vaak gevoelige informatie, zoals persoonsgegevens of contractdetails. Dit verhoogt het risico op datalekken en misbruik, zeker wanneer onvoldoende controle bestaat over de inhoud en het verloop van deze gesprekken \autocite{Bender2021}.\\

\noindent Daarnaast blijkt uit recent onderzoek dat AI-systemen kwetsbaar zijn voor specifieke aanvalsvormen, zoals prompt-injecties en data-exfiltratie via doelbewust gemanipuleerde input \autocite{Greshake2023}. Deze kwetsbaarheden tonen aan dat klassieke beveiligingsmaatregelen niet volstaan om AI-gedreven interacties voldoende te beschermen.

\subsection{Beperkingen van traditionele beveiligingsmechanismen}

\noindent Traditionele web- en API-beveiliging steunt voornamelijk op statische regels en vooraf gedefinieerde patronen. Firewalls en intrusion detection systems zijn effectief tegen bekende aanvalsvectoren, maar hebben moeite om contextuele en semantische aanvallen te detecteren \autocite{Sommer2010}. In het geval van AI-chatbots, waar input en output contextafhankelijk zijn, leidt dit tot een verhoogd risico op false negatives en onvoldoende bescherming.\\

\noindent Om deze beperkingen tegen te gaan, worden steeds vaker AI-gestuurde beveiligingsoplossingen voorgesteld. Akamai beschrijft in zijn recente publicaties hoe machine learning kan worden ingezet om afwijkende patronen in web- en API-verkeer te herkennen en dynamisch te mitigeren \autocite{AkamaiFirewallAI2025}. Dergelijke oplossingen bieden nieuwe mogelijkheden, maar vereisen een integratie van observability- en auditsysteem om hun effectiviteit te kunnen evalueren.

\subsection{Observability en auditability van AI-systemen}

\noindent Observability verwijst naar het vermogen om de interne toestand van een systeem te reconstrueren op basis van externe signalen zoals logs, metrics en traces. In moderne, complexe en gedistribueerde omgevingen is observability cruciaal om systeemgedrag te analyseren, incidenten te onderzoeken en performantie te optimaliseren \autocite{Li2021}. Dynatrace benadrukt dat observability steeds belangrijker wordt in AI-gedreven toepassingen, omdat de besluitvorming van modellen vaak complex is en moeilijk te verklaren valt \autocite{DynatraceAIObs2025}.\\


\noindent Audit logging speelt hierbij een centrale rol. Door interacties, systeembeslissingen en beveiligingsevents systematisch te registreren, wordt het mogelijk om AI-gedreven processen achteraf te reconstrueren en te analyseren. Dergelijke auditmechanismen zijn essentieel voor troubleshooting, maar ook voor het aantonen van verantwoordelijkheid en naleving van regelgevingen \autocite{Novelli2023}. Recente onderzoeken laten zien dat audit logging bij AI-systemen vaak los van de rest wordt toegepast. Er is meestal geen goede koppeling tussen wat je kunt zien in observability-data en de security-events, waardoor het lastiger wordt om achteraf precies te begrijpen wat er gebeurd is of incidenten goed te analyseren.

\subsection{Privacy en regelgeving}

\noindent Het loggen van chatbotinteracties brengt onvermijdelijk privacyvraagstukken met zich mee. De General Data Protection Regulation (GDPR) legt strikte voorwaarden op voor het verzamelen, verwerken en bewaren van persoonsgegevens. Volgens \textcite{Voigt2017} moeten systemen die persoonsgegevens verwerken voldoen aan het principe van privacy by design, waarbij dataminimalisatie en transparantie centraal \\staan.\\

\noindent In het geval van AI-chatbots betekent dit dat logging en tracing zorgvuldig moeten worden ontworpen om gevoelige gegevens te beschermen, bijvoorbeeld via pseudonimisering of selective logging \autocite{EDPBGuidelines4_2019}. De tekst benadrukt dat veiligheid en privacy geen tegengestelde doelen zijn, maar elkaar kunnen versterken wanneer ze vanaf het ontwerp geïntegreerd worden.


%---------- Methodologie ------------------------------------------------------
\section{Methodologie}%
\label{sec:methodologie}

\begin{figure*}[h]
    \includegraphics[width=\textwidth]{gantt.png}
    \caption{Gantt-diagram}
\end{figure*}

\noindent Dit onderzoek wordt uitgevoerd als toegepast onderzoek binnen de bedrijfscontext van Evolane en focust op één afgebakende probleemsituatie, namelijk het gebrek aan traceerbaarheid en beveiliging van AI-gestuurde chatbotinteracties binnen klantondersteuningsprocessen. Het onderzoek resulteert in een minimum viable product (MVP), zijnde een eerste functionele implementatie waarin de kernfunctionaliteiten voor traceerbaarheid en beveiliging aantoonbaar aanwezig zijn. De scope van het onderzoek beperkt zich tot het technisch ontwerpen, implementeren en evalueren van dit prototype binnen een gecontroleerde testomgeving.\\

\noindent Het proof-of-concept wordt opgebouwd in vier opeenvolgende, inhoudelijk samenhangende fasen. Elke fase levert een concrete mijlpaal op die bijdraagt aan het beantwoorden van de centrale onderzoeksvraag.

\paragraph{Fase 1: Ontwikkeling van een afgebakende chatbot}
In de eerste fase wordt een eenvoudige chatbot ontwikkeld die één concreet klantondersteuningsscenario implementeert. De functionaliteit blijft bewust beperkt tot een duidelijk afgelijnde use case, zodat de interactiestroom tussen gebruiker en digitale assistent reproduceerbaar blijft. De chatbot dient uitsluitend als testobject voor verdere beveiligings- en traceerbaarheidsmechanismen. Het eindresultaat van deze fase is een functionerende chatbot met een stabiele interactiestroom.

\paragraph{Fase 2: AI firewall voor chatbotverkeer}
In de tweede fase wordt het chatbotverkeer voorzien van een beveiligingslaag die inkomende en uitgaande interacties analyseert. De focus ligt op het detecteren en blokkeren van afwijkende of risicovolle invoer binnen de afgebakende testomgeving. Door het uitvoeren van gesimuleerde aanvalsscenario’s wordt nagegaan hoe effectief het systeem reageert op misbruik van de chatbotinterface. Deze fase resulteert in een chatbot waarbij beveiligingsdetectie en -mitigatie functioneren.

\paragraph{Fase 3: Implementatie van traceerbaarheid en audit trail}
De derde fase richt zich op het vastleggen en overeenkomen van gebeurtenissen binnen de chatbotinteracties. Hierbij worden gebruikersacties, systeemreacties en beveiligingsgebeurtenissen gelogd om volledige reconstructie van gesprekken mogelijk te maken. De audit trail wordt opgezet met als doel technische reproduceerbaarheid van incidenten, zonder dat juridische bewijsvoering of langdurige datastorage wordt nagestreefd. Het eindresultaat van deze fase is een werkende audit trail die inzicht biedt in het volledige verloop van interacties.

\paragraph{Fase 4: Integratie en evaluatie van het MVP}
In de vierde fase worden alle ontwikkelde componenten samengebracht tot één geïntegreerd MVP. Dit prototype wordt geëvalueerd aan de hand van vooraf gedefinieerde testscenario’s die zowel normale interacties als afwijkend gedrag omvatten. De evaluatie focust op de mate waarin het systeem traceerbaarheid en beveiliging combineert binnen de afgebakende casus. Deze fase resulteert in een geïntegreerd MVP en een technische evaluatie die de basis vormt voor aanbevelingen.


%---------- Verwachte resultaten ----------------------------------------------
\section{Verwacht resultaat, conclusie}%
\label{sec:verwachte_resultaten}

\noindent Het voornaamste verwachte resultaat van deze bachelorproef is een functionerend proof-of-concept van een geïntegreerd framework waarin een AI-chatbot, een beveiligingslaag en een audit trail samenkomen tot een minimum viable product (MVP). Dit MVP zal aantonen dat interacties tussen eindgebruikers en de chatbot traceerbaar, reproduceerbaar en beveiligd kunnen worden, zonder dat gevoelige informatie onnodig wordt blootgesteld.\\

\noindent Specifiek wordt verwacht dat het prototype de volgende kenmerken vertoont:
\begin{itemize}
\item Traceerbaarheid van gesprekken: alle prompts, antwoorden, sessie-ID’s en tijdstempels worden gelogd en zijn reconstructeerbaar.

\item Detectie en mitigatie van risicovolle input: de beveiligingslaag kan verdachte of afwijkende gebruikersinvoer herkennen en blokkeren of markeren voor nadere analyse.

\item Gecorreleerde observability: beveiligingsincidenten en interactiegegevens zijn verbonden in een audit trail, waardoor een volledig overzicht van elk gesprek en mogelijke incidenten ontstaat.

\end{itemize}

\noindent Het onderzoek biedt inzicht in de praktische haalbaarheid van het combineren van observability en AI-gestuurde beveiliging. Verwacht wordt dat de resultaten aantonen welke componenten effectief bijdragen aan veiligheid en traceerbaarheid, en waar nog optimalisaties mogelijk zijn.



\printbibliography[heading=bibintoc]

\end{document}